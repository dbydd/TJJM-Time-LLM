% \vspace{-2mm}
\revision{\section{Few-shot and Zero-shot Forecasting}\label{appx:zero-shot}}
\vspace{-10mm}
\revision{\subsection{Few-shot Forecasting}}
\begin{table}[b!]
\captionsetup{font=small} 
\caption{\revision{Full few-shot learning results on 10\% training data. We use the same protocol as in \shortautoref{tab:long-term-forecasting}.}}
\label{tab:few-shot-forecasting-10per-full}
\begin{center}
\begin{small}
\scalebox{0.60}{
\setlength\tabcolsep{3pt}
\begin{tabular}{c|c|cc|cc|cc|cc|cc|cc|cc|cc|cc|cc|cc|cc}
\toprule

\multicolumn{2}{c|}{Methods}&\multicolumn{2}{c|}{\method{}}&\multicolumn{2}{c|}{GPT4TS}&\multicolumn{2}{c|}{DLinear}&\multicolumn{2}{c|}{PatchTST}&\multicolumn{2}{c|}{TimesNet}&\multicolumn{2}{c|}{FEDformer}&\multicolumn{2}{c|}{Autoformer}&\multicolumn{2}{c|}{Stationary}&\multicolumn{2}{c|}{ETSformer}&\multicolumn{2}{c|}{LightTS}&\multicolumn{2}{c|}{Informer}&\multicolumn{2}{c}{Reformer} \\

\midrule

\multicolumn{2}{c|}{Metric} & MSE  & MAE & MSE & MAE& MSE & MAE& MSE  & MAE& MSE  & MAE& MSE  & MAE& MSE  & MAE& MSE  & MAE& MSE  & MAE& MSE  & MAE& MSE  & MAE& MSE  & MAE\\
\midrule

\multirow{5}{*}{\rotatebox{90}{$ETTh1$}}
& 96  &\boldres{0.448} &\secondres{0.460} & \secondres{0.458} & \boldres{0.456} & 0.492 & 0.495 & 0.516 & 0.485 & 0.861& 0.628& 0.512 & 0.499 & 0.613 & 0.552 &0.918&0.639& 1.112 & 0.806 & 1.298 & 0.838 & 1.179 & 0.792 &1.184&0.790\\
& 192 &\boldres{0.484} &\boldres{0.483} & 0.570 & \secondres{0.516} & \secondres{0.565} & 0.538 & 0.598 & 0.524 & 0.797& 0.593& 0.624 & 0.555 & 0.722 & 0.598 &0.915&0.629& 1.155 & 0.823 & 1.322 & 0.854 & 1.199 & 0.806 &1.295&0.850\\
& 336 & \boldres{0.589} & \secondres{0.540} & \secondres{0.608} & \boldres{0.535} & 0.721 & 0.622 & 0.657 & 0.550 & 0.941& 0.648& 0.691 & 0.574 & 0.750 & 0.619 &0.939&0.644& 1.179 & 0.832 & 1.347 & 0.870 & 1.202 & 0.811 &1.294&0.854\\
& 720 &\boldres{0.700} &\boldres{0.604} & \secondres{0.725} &\secondres{0.591} & 0.986 & 0.743 & 0.762 & 0.610 &0.877 &0.641 & 0.728 & 0.614 & 0.721 & 0.616 &0.887&0.645& 1.273 & 0.874 & 1.534 & 0.947 & 1.217 & 0.825 &1.223&0.838\\
&Avg &\boldres{0.556} &\boldres{0.522} &\secondres{0.590} &\secondres{0.525} &0.691 &0.600 &0.633 &0.542 &0.869 &0.628 &0.639 &0.561 &0.702 &0.596 &0.915 &0.639 &1.180 &0.834 &1.375 &0.877 &1.199 &0.809 &1.249 &0.833\\
\midrule

\multirow{5}{*}{\rotatebox{90}{$ETTh2$}}
& 96  &\boldres{0.275} &\boldres{0.326} & \secondres{0.331} & \secondres{0.374} & 0.357 & 0.411 & 0.353 & 0.389 & 0.378& 0.409& 0.382 & 0.416 & 0.413 & 0.451 &0.389&0.411& 0.678 & 0.619 & 2.022 & 1.006 & 3.837 & 1.508&3.788&1.533\\
& 192 &\boldres{0.374} &\boldres{0.373} &\secondres{ 0.402} & \secondres{0.411} & 0.569 & 0.519 & 0.403 & 0.414 & 0.490& 0.467& 0.478 & 0.474 & 0.474 & 0.477 &0.473&0.455& 0.785 & 0.666 & 2.329 & 1.104 & 3.856 & 1.513 &3.552&1.483\\
& 336 &\boldres{0.406} &\boldres{0.429} & \boldres{0.406} & \secondres{0.433} & 0.671 & 0.572 & \secondres{0.426} & 0.441 & 0.537& 0.494& 0.504 & 0.501 & 0.547 & 0.543 &0.507&0.480& 0.839 & 0.694 & 2.453 & 1.122 & 3.952 & 1.526 &3.395&1.526\\
& 720 &\boldres{0.427} &\boldres{0.449} & \secondres{0.449} & \secondres{0.464} & 0.824 & 0.648 & 0.477 & 0.480 & 0.510& 0.491& 0.499 & 0.509 & 0.516 & 0.523 &0.477&0.472& 1.273 & 0.874 & 3.816 & 1.407 & 3.842 & 1.503 &3.205&1.401\\
&Avg &\boldres{0.370} &\boldres{0.394} &\secondres{0.397} &\secondres{0.421} &0.605 &0.538 &0.415 &0.431 &0.479 &0.465 &0.466 &0.475 &0.488 &0.499 &0.462 &0.455 &0.894 &0.713 &2.655 &1.160 &3.872 &1.513 &3.485 &1.486\\
\midrule

\multirow{5}{*}{\rotatebox{90}{$ETTm1$}}
& 96  & \boldres{0.346} &\boldres{0.388} & 0.390 & 0.404 & \secondres{0.352} & 0.392 & 0.410 & 0.419 & 0.583& 0.501& 0.578 & 0.518 & 0.774 & 0.614 &0.761&0.568& 0.911 & 0.688 & 0.921 & 0.682 & 1.162 & 0.785&1.442&0.847\\
& 192 &\boldres{0.373} &\secondres{0.416} & 0.429 & 0.423 & \secondres{0.382} & \boldres{0.412} & 0.437 & 0.434 & 0.630& 0.528& 0.617 & 0.546 & 0.754 & 0.592 &0.781&0.574& 0.955 & 0.703 & 0.957 & 0.701 & 1.172 & 0.793 &1.444&0.862\\
& 336 &\boldres{0.413} &\boldres{0.426} & 0.469 & 0.439 &\secondres{ 0.419} &\secondres{0.434} & 0.476 & 0.454 & 0.725& 0.568& 0.998 & 0.775 & 0.869 & 0.677 &0.803&0.587& 0.991 & 0.719 & 0.998 & 0.716 & 1.227 & 0.908&1.450&0.866 \\
& 720 &\boldres{0.485} &\secondres{0.476} & 0.569 & 0.498 & \secondres{0.490} & \boldres{0.477} & 0.681 & 0.556 & 0.769& 0.549& 0.693 & 0.579 & 0.810 & 0.630 &0.844&0.581& 1.062 & 0.747 & 1.007 & 0.719 & 1.207 & 0.797&1.366&0.850 \\
&Avg &\boldres{0.404} &\boldres{0.427} &0.464 &0.441 &\secondres{0.411} &\secondres{0.429} &0.501 &0.466 &0.677 &0.537 &0.722 &0.605 &0.802 &0.628 &0.797 &0.578 &0.980 &0.714 &0.971 &0.705 &1.192 &0.821 &1.426 &0.856\\
\midrule

\multirow{5}{*}{\rotatebox{90}{$ETTm2$}}
& 96  &\boldres{0.177} &\boldres{0.261} &\secondres{0.188} & \secondres{0.269}  & 0.213 & 0.303 & 0.191 & 0.274 & 0.212& 0.285& 0.291 & 0.399 & 0.352 & 0.454 &0.229&0.308& 0.331 & 0.430 & 0.813 & 0.688 & 3.203 & 1.407 &4.195&1.628\\
& 192 &\boldres{0.241} &\boldres{0.314} & \secondres{0.251} & \secondres{0.309} & 0.278 & 0.345 & 0.252 & 0.317 & 0.270& 0.323& 0.307 & 0.379 & 0.694 & 0.691 &0.291&0.343& 0.400 & 0.464 & 1.008 & 0.768 & 3.112 & 1.387&4.042&1.601 \\
& 336 &\boldres{0.274} &\boldres{0.327}& 0.307 &\secondres{0.346} & 0.338 & 0.385 & \secondres{0.306} & 0.353 & 0.323& 0.353& 0.543 & 0.559 & 2.408 & 1.407 &0.348&0.376& 0.469 & 0.498 & 1.031 & 0.775 & 3.255 & 1.421&3.963&1.585 \\
& 720 &\boldres{0.417} &\boldres{0.390} &\secondres{0.426} &\secondres{0.417} & 0.436 & 0.440 & 0.433 & 0.427 & 0.474& 0.449& 0.712 & 0.614 & 1.913 & 1.166 &0.461&0.438& 0.589 & 0.557 & 1.096 & 0.791 & 3.909 & 1.543&3.711&1.532 \\
&Avg &\boldres{0.277} &\boldres{0.323} &\secondres{0.293} &\secondres{0.335} &0.316 &0.368 &0.296 &0.343 &0.320 &0.353 &0.463 &0.488 &1.342 &0.930 &0.332 &0.366 &0.447 &0.487 &0.987 &0.756 &3.370 &1.440 &3.978 &1.587\\
\midrule

\multirow{5}{*}{\rotatebox{90}{$\revision{Weather}$}}
& \revision{96}  &\boldres{0.161} &\boldres{0.210} &\secondres{0.163} &\secondres{0.215} &0.171 &0.224 &0.165 &0.215 &0.184 &0.230 &0.188 &0.253 &0.221 &0.297 &0.192 &0.234 &0.199 &0.272 &0.217 &0.269 &0.374 &0.401 &0.335 &0.380 \\
& \revision{192}  &\boldres{0.204} &\boldres{0.248} &\secondres{0.210} &\secondres{0.254} &0.215 &0.263 &0.210 &0.257 &0.245 &0.283 &0.250 &0.304 &0.270 &0.322 &0.269 &0.295 &0.279 &0.332 &0.259 &0.304 &0.552 &0.478 &0.522 &0.462 \\
& \revision{336}  &0.261 &0.302 &\boldres{0.256} &\boldres{0.292} &\secondres{0.258} &0.299 &0.259 &\secondres{0.297} &0.305 &0.321 &0.312 &0.346 &0.320 &0.351 &0.370 &0.357 &0.356 &0.386 &0.303 &0.334 &724 &0.541 &0.715 &0.535 \\
& \revision{720}  &\boldres{0.309} &\boldres{0.332} &0.321 &\secondres{0.339} &\secondres{0.320} &0.346 &0.332 &0.346 &0.381 &0.371 &0.387 &0.393 &0.390 &0.396 &0.441 &0.405 &0.437 &0.448 &0.377 &0.382 &0.739 &0.558 &0.611 &0.500 \\
& \revision{Avg}  &\boldres{0.234} &\boldres{0.273} &\secondres{0.238} &\secondres{0.275} &0.241 &0.283 &0.242 &0.279 &0.279 &0.301 &0.284 &0.324 &0.300 &0.342 &0.318 &0.323 &0.318 &0.360 &0.289 &0.322 &0.597 &0.495 &0.546 &0.469 \\
\midrule

\multirow{5}{*}{\rotatebox{90}{$\revision{Electricity}$}}
& \revision{96}  &\boldres{0.139} &0.241 &\boldres{0.139} &\boldres{0.237} &0.150 &0.253 &\secondres{0.140} &\secondres{0.238} &0.299 &0.373 &0.231 &0.323 &0.261 &0.348 &0.420 &0.466 &0.599 &0.587 &0.350 &0.425 &1.259 &0.919 &0.993 &0.784 \\
& \revision{192}  &\boldres{0.151} &\boldres{0.248} &\secondres{0.156} &\secondres{0.252} &0.164 &0.264 &0.160 &0.255 &0.305 &0.379 &0.261 &0.356 &0.338 &0.406 &0.411 &0.459 &0.620 &0.598 &0.376 &0.448 &1.160 &0.873 &0.938 &0.753 \\
& \revision{336}  &\boldres{0.169} &\boldres{0.270} &\secondres{0.175} &\boldres{0.270} &0.181 &0.282 &0.180 &0.276 &0.319 &0.391 &0.360 &0.445 &0.410 &0.474 &0.434 &0.473 &0.662 &0.619 &0.428 &0.485 &1.157 &0.872 &0.925 &0.745 \\
& \revision{720}  &0.240 &0.322 &\secondres{0.233 }&\boldres{0.317} &\boldres{0.223} &\secondres{0.321} &0.241 &0.323 &0.369 &0.426 &0.530 &0.585 &0.715 &0.685 &0.510 &0.521 &0.757 &0.664 &0.611 &0.597 &1.203 &0.898 &1.004 &0.790 \\
& \revision{Avg}  &\boldres{0.175} &\secondres{0.270} &\secondres{0.176} &\boldres{0.269} &0.180 &0.280 &0.180 &0.273 &0.323 &0.392 &0.346 &0.427 &0.431 &0.478 &0.444 &0.480 &0.660 &0.617 &0.441 &0.489 &1.195 &0.891 &0.965 &0.768 \\
\midrule

\multirow{5}{*}{\rotatebox{90}{$\revision{Traffic}$}}
& \revision{96}  &0.418 &\secondres{0.291} &\secondres{0.414} &0.297 &0.419 &0.298 &\boldres{0.403} &\boldres{0.289} &0.719 &0.416 &0.639 &0.400 &0.672 &0.405 &1.412 &0.802 &1.643 &0.855 &1.157 &0.636 &1.557 &0.821 &1.527 &0.815 \\
& \revision{192}  &\boldres{0.414} &\boldres{0.296} &0.426 &\secondres{0.301} &0.434 &0.305 &\secondres{0.415} &\boldres{0.296} &0.748 &0.428 &0.637 &0.416 &0.727 &0.424 &1.419 &0.806 &1.641 &0.854 &1.207 &0.661 &1.454 &0.765 &1.538 &0.817 \\
& \revision{336}  &\boldres{0.421} &0.311 &\secondres{0.434} &\boldres{0.303} &0.449 &0.313 &0.426 &\secondres{0.304} &0.853 &0.471 &0.655 &0.427 &0.749 &0.454 &1.443 &0.815 &1.711 &0.878 &1.334 &0.713 &1.521 &0.812 &1.550 &0.819 \\
& \revision{720}  &\boldres{0.462} &\boldres{0.327} &0.487 &0.337 &0.484 &0.336 &\secondres{0.474} &\secondres{0.331} &1.485 &0.825 &0.722 &0.456 &0.847 &0.499 &1.539 &0.837 &2.660 &1.157 &1.292 &0.726 &1.605 &0.846 &1.588 &0.833 \\
& \revision{Avg}  &\boldres{0.429} &\secondres{0.306} &0.440 &0.310 &0.447 &0.313 &\secondres{0.430} &\boldres{0.305} &0.951 &0.535 &0.663 &0.425 &0.749 &0.446 &1.453 &0.815 &1.914 &0.936 &1.248 &0.684 &1.534 &0.811 &1.551 &0.821 \\
\midrule

\multicolumn{2}{c|}{$1^{\text{st}}$Count}&\multicolumn{2}{c|}{\boldres{32}}&\multicolumn{2}{c|}{\secondres{9}}&\multicolumn{2}{c|}{3}&\multicolumn{2}{c|}{3}&\multicolumn{2}{c|}{0}&\multicolumn{2}{c|}{0}&\multicolumn{2}{c|}{0}&\multicolumn{2}{c|}{0}&\multicolumn{2}{c|}{0}&\multicolumn{2}{c|}{0}&\multicolumn{2}{c|}{0}&\multicolumn{2}{c}{0}\\
\bottomrule
\end{tabular}
}
\end{small}
\end{center}
\vskip -0.1in
\end{table}
\begin{table}[h!]
\captionsetup{font=small} 
\caption{\revision{Full few-shot learning results on 5\% training data. We use the same protocol as in \shortautoref{tab:long-term-forecasting}. '-' means that 5\% time series is not sufficient to constitute a training set.}}
\label{tab:few-shot-forecasting-5per-full}
\begin{center}
\begin{small}
\scalebox{0.60}{
\setlength\tabcolsep{3pt}
\begin{tabular}{c|c|cc|cc|cc|cc|cc|cc|cc|cc|cc|cc|cc|cc}
\toprule

\multicolumn{2}{c|}{Methods}&\multicolumn{2}{c|}{\method}&\multicolumn{2}{c|}{GPT4TS}&\multicolumn{2}{c|}{DLinear}&\multicolumn{2}{c|}{PatchTST}&\multicolumn{2}{c|}{TimesNet}&\multicolumn{2}{c|}{FEDformer}&\multicolumn{2}{c|}{Autoformer}&\multicolumn{2}{c|}{Stationary}&\multicolumn{2}{c|}{ETSformer}&\multicolumn{2}{c|}{LightTS}&\multicolumn{2}{c|}{Informer}&\multicolumn{2}{c}{Reformer} \\

\midrule

\multicolumn{2}{c|}{Metric} & MSE  & MAE & MSE & MAE& MSE & MAE& MSE  & MAE& MSE  & MAE& MSE  & MAE& MSE  & MAE& MSE  & MAE& MSE  & MAE& MSE  & MAE& MSE  & MAE& MSE  & MAE\\

\midrule

\multirow{5}{*}{\rotatebox{90}{$ETTh1$}}
& 96  &\boldres{0.483} &\boldres{0.464} & \secondres{0.543} & 0.506 & 0.547 & \secondres{0.503} & 0.557 & 0.519 & 0.892& 0.625& 0.593 & 0.529 & 0.681 & 0.570 &0.952&0.650& 1.169 & 0.832 & 1.483 & 0.91 & 1.225 & 0.812 &1.198&0.795 \\
& 192 &\boldres{0.629} &\boldres{0.540} & 0.748 & 0.580 & 0.720 & 0.604 & 0.711 & 0.570 &0.940 & 0.665& \secondres{0.652} & \secondres{0.563} & 0.725 & 0.602 &0.943&0.645& 1.221 & 0.853 & 1.525 & 0.93 & 1.249 & 0.828 &1.273&0.853\\
& 336 &0.768 &0.626 & \secondres{0.754} & \secondres{0.595} & 0.984 & 0.727 & 0.816 & 0.619 & 0.945&0.653 & \boldres{0.731} & \boldres{0.594} & 0.761 & 0.624 &0.935&0.644& 1.179 & 0.832 & 1.347 & 0.87 & 1.202 & 0.811 &1.254&0.857 \\
& 720 & - & - & - & - & - & - & - & - & - & - & - & - & - & - & - & - & - & - & - & - & - & - & - & - \\
&Avg &\boldres{0.627} &\boldres{0.543}&0.681&\secondres{0.560}&0.750&0.611&0.694&0.569&0.925&0.647&\secondres{0.658}&0.562&0.722&0.598&0.943&0.646&1.189&0.839&1.451&0.903&1.225&0.817&1.241&0.835\\
\midrule


\multirow{5}{*}{\rotatebox{90}{$ETTh2$}}
& 96  &\boldres{0.336} &\boldres{0.397} & \secondres{0.376} & \secondres{0.421} & 0.442 & 0.456 & 0.401 & 0.421 & 0.409& 0.420& 0.390 & 0.424 & 0.428 & 0.468 &0.408&0.423 & 0.678 & 0.619 & 2.022 & 1.006 & 3.837 & 1.508 &3.753&1.518\\
& 192 &\boldres{0.406} &\boldres{0.425} &\secondres{0.418} &\secondres{0.441} & 0.617 & 0.542 & 0.452 & 0.455 & 0.483& 0.464& 0.457 & 0.465 & 0.496 & 0.504 &0.497&0.468 & 0.845 & 0.697 & 3.534 & 1.348 & 3.975 & 1.933 &3.516&1.473\\
& 336 &\boldres{0.405} &\boldres{0.432} & 0.408 & 0.439 & 1.424 & 0.849 & 0.464 & 0.469 & 0.499& 0.479& 0.477 & 0.483 & 0.486 & 0.496 &0.507&0.481 & 0.905 & 0.727 & 4.063 & 1.451 & 3.956 & 1.520 &3.312&1.427\\
& 720 & - & - & - & - & - & - & - & - & - & - & - & - & - & - & - & - & - & - & - & - & - & - & - & - \\
&Avg &\boldres{0.382} &\boldres{0.418} &\secondres{0.400}&\secondres{0.433}&0.694&0.577&0.827&0.615&0.439&0.448&0.463&0.454&0.441&0.457&0.470&0.489&0.809&0.681&3.206&1.268&3.922&1.653&3.527&1.472\\
\midrule

\multirow{5}{*}{\rotatebox{90}{$ETTm1$}}
& 96  &\boldres{0.316} &\secondres{0.377} & 0.386 & 0.405 &\secondres{0.332} &\boldres{0.374} & 0.399 & 0.414 & 0.606& 0.518& 0.628 & 0.544 & 0.726 & 0.578 &0.823&0.587 & 1.031 & 0.747 & 1.048 & 0.733 & 1.130 & 0.775 &1.234&0.798\\
& 192 &0.450 &0.464 &\boldres{0.440} & \secondres{0.438} & 0.358 & 0.390 & \secondres{0.441} &\boldres{0.436 }& 0.681& 0.539& 0.666 & 0.566 & 0.750 & 0.591 &0.844&0.591 & 1.087 & 0.766 & 1.097 & 0.756 & 1.150 & 0.788 &1.287&0.839\\
& 336 &\secondres{0.450} &\secondres{0.424} & 0.485 & 0.459 & \boldres{0.402} &\boldres{0.416} & 0.499 & 0.467 & 0.786& 0.597& 0.807 & 0.628 & 0.851 & 0.659 &0.870&0.603 & 1.138 & 0.787 & 1.147 & 0.775 & 1.198 & 0.809 &1.288&0.842\\
& 720 &\boldres{0.483} &\boldres{0.471} & 0.577 & 0.499 & \secondres{0.511} & \secondres{0.489} & 0.767 & 0.587 & 0.796& 0.593& 0.822 & 0.633 & 0.857 & 0.655 &0.893&0.611 & 1.245 & 0.831 & 1.200 & 0.799 & 1.175 & 0.794 &1.247&0.828\\
&Avg &\secondres{0.425} &\secondres{0.434} &0.472&0.450&\boldres{0.400}&\boldres{0.417}&0.526&0.476&0.717&0.561&0.730&0.592&0.796&0.620&0.857&0.598&1.125&0.782&1.123&0.765&1.163&0.791&1.264&0.826\\
\midrule

\multirow{5}{*}{\rotatebox{90}{$ETTm2$}}
& 96  &\boldres{0.174} &\boldres{0.261} &\secondres{0.199} &\secondres{0.280} & 0.236 & 0.326 & 0.206 & 0.288 & 0.220& 0.299& 0.229 & 0.320 & 0.232 & 0.322 &0.238&0.316 & 0.404 & 0.485 & 1.108 & 0.772 & 3.599 & 1.478&3.883&1.545 \\
& 192 &\boldres{0.215} &\boldres{0.287} & \secondres{0.256} & \secondres{0.316} & 0.306 & 0.373 & 0.264 & 0.324 & 0.311& 0.361& 0.394 & 0.361 & 0.291 & 0.357 &0.298&0.349 & 0.479 & 0.521 & 1.317 & 0.850 & 3.578 & 1.475 &3.553&1.484\\
& 336 &\boldres{0.273} &\boldres{0.330} &\secondres{0.318} &\secondres{0.353} & 0.380 & 0.423 & 0.334 & 0.367 & 0.338& 0.366& 0.378 & 0.427 & 0.478 & 0.517 &0.353&0.380 & 0.552 & 0.555 & 1.415 & 0.879 & 3.561 & 1.473 &3.446&1.460\\
& 720 &\boldres{0.433} &\boldres{0.412} & 0.460 & 0.436 & 0.674 & 0.583 & \secondres{0.454} &\secondres{0.432} & 0.509& 0.465& 0.523 & 0.510 & 0.553 & 0.538 &0.475&0.445 & 0.701 & 0.627 & 1.822 & 0.984 & 3.896 & 1.533 &3.445&1.460\\
&Avg &\boldres{0.274} &\boldres{0.323}&\secondres{0.308}&\secondres{0.346}&0.399&0.426&0.314&0.352&0.344&0.372&0.381&0.404&0.388&0.433&0.341&0.372&0.534&0.547&1.415&0.871&3.658&1.489&3.581&1.487\\
\midrule

\multirow{5}{*}{\rotatebox{90}{$\revision{Weather}$}}
& \revision{96}  &\secondres{0.172} &0.263 &0.175 &\secondres{0.230} &0.184 &0.242 &\boldres{0.171} &\boldres{0.224} &0.207 &0.253 &0.229 &0.309 &0.227 &0.299 &0.215 &0.252 &0.218 &0.295 &0.230 &0.285 &0.497 &0.497 &0.406 &0.435 \\
& \revision{192}  &\boldres{0.224} &\boldres{0.271} &\secondres{0.227} &\secondres{0.276} &0.228 &0.283 &0.230 &0.277 &0.272 &0.307 &0.265 &0.317 &0.278 &0.333 &0.290 &0.307 &0.294 &0.331 &0.274 &0.323 &0.620 &0.545 &0.446 &0.450 \\
& \revision{336}  &\secondres{0.282} &\boldres{0.321} &0.286 &\secondres{0.322} &\boldres{0.279} &\secondres{0.322} &0.294 &0.326 &0.313 &0.328 &0.353 &0.392 &0.351 &0.393 &0.353 &0.348 &0.359 &0.398 &0.318 &0.355 &0.649 &0.547 &0.465 &0.459 \\
& \revision{720}  &\secondres{0.366} &\secondres{0.381} &\secondres{0.366} &\boldres{0.379} &\boldres{0.364} &0.388 &0.384 &0.387 &0.400 &0.385 &0.391 &0.394 &0.387 &0.389 &0.452 &0.407 &0.461 &0.461 &0.401 &0.418 &0.570 &0.522 &0.471 &0.468 \\
& \revision{Avg}  &\boldres{0.260} &0.309 &\secondres{0.263} &\boldres{0.301} &\secondres{0.263} &0.308 &0.269 &\secondres{0.303} &0.298 &0.318 &0.309 &0.353 &0.310 &0.353 &0.327 &0.328 &0.333 &0.371 &0.305 &0.345 &0.584 &0.527 &0.447 &0.453 \\
\midrule

\multirow{5}{*}{\rotatebox{90}{$\revision{Electricity}$}}
& \revision{96}  &0.147 &\secondres{0.242} &\boldres{0.143} &\boldres{0.241} &0.150 &0.251 &\secondres{0.145} &0.244 &0.315 &0.389 &0.235 &0.322 &0.297 &0.367 &0.484 &0.518 &0.697 &0.638 &0.639 &0.609 &1.265 &0.919 &1.414 &0.855 \\
& \revision{192}  &\boldres{0.158} &\boldres{0.241} &\secondres{0.159 }&\secondres{0.255} &0.163 &0.263 &0.163 &0.260 &0.318 &0.396 &0.247 &0.341 &0.308 &0.375 &0.501 &0.531 &0.718 &0.648 &0.772 &0.678 &1.298 &0.939 &1.240 &0.919 \\
& \revision{336}  &\secondres{0.178} &\secondres{0.277} &0.179 &\boldres{0.274} &\boldres{0.175} &0.278 &0.183 &0.281 &0.340 &0.415 &0.267 &0.356 &0.354 &0.411 &0.574 &0.578 &0.758 &0.667 &0.901 &0.745 &1.302 &0.942 &1.253 &0.921 \\
& \revision{720}  &\secondres{0.224} &\secondres{0.312} &0.233 &0.323 &\boldres{0.219} &\boldres{0.311} &0.233 &0.323 &0.635 &0.613 &0.318 &0.394 &0.426 &0.466 &0.952 &0.786 &1.028 &0.788 &1.200 &0.871 &1.259 &0.919 &1.249 &0.921 \\
& \revision{Avg}  &\secondres{0.179} &\boldres{0.268} &\boldres{0.178} &\secondres{0.273} &0.176 &0.275 &0.181 &0.277 &0.402 &0.453 &0.266 &0.353 &0.346 &0.404 &0.627 &0.603 &0.800 &0.685 &0.878 &0.725 &1.281 &0.929 &1.289 &0.904 \\
\midrule

\multirow{5}{*}{\rotatebox{90}{$\revision{Traffic}$}}
& \revision{96}  &\secondres{0.414} &\secondres{0.291} &0.419 &0.298 &0.427 &0.304 &\boldres{0.404} &\boldres{0.286} &0.854 &0.492 &0.670 &0.421 &0.795 &0.481 &1.468 &0.821 &1.643 &0.855 &1.157 &0.636 &1.557 &0.821 &1.586 &0.841 \\
& \revision{192}  &\secondres{0.419} &\boldres{0.291} &0.434 &0.305 &0.447 &0.315 &\boldres{0.412} &\secondres{0.294} &0.894 &0.517 &0.653 &0.405 &0.837 &0.503 &1.509 &0.838 &1.856 &0.928 &1.688 &0.848 &1.596 &0.834 &1.602 &0.844 \\
& \revision{336}  &\boldres{0.437} &\secondres{0.314} &0.449 &0.313 &0.478 &0.333 &\secondres{0.439} &\boldres{0.310} &0.853 &0.471 &0.707 &0.445 &0.867 &0.523 &1.602 &0.860 &2.080 &0.999 &1.826 &0.903 &1.621 &0.841 &1.668 &0.868 \\
& \revision{720}  &- &- &- &- &- &- &- &- &- &- &- &- &- &- &- &- &- &- &- &- &- &- &- &- \\
& \revision{Avg}  &\secondres{0.423} &\secondres{0.298} &0.434 &0.305 &0.450 &0.317 &\boldres{0.418} &\boldres{0.296} &0.867 &0.493 &0.676 &0.423 &0.833 &0.502 &1.526 &0.839 &1.859 &0.927 &1.557 &0.795 &1.591 &0.832 &1.618 &0.851 \\
\midrule

\multicolumn{2}{c|}{$1^{\text{st}}$Count}&\multicolumn{2}{c|}{\boldres{21}}&\multicolumn{2}{c|}{6}&\multicolumn{2}{c|}{\secondres{7}}&\multicolumn{2}{c|}{6}&\multicolumn{2}{c|}{0}&\multicolumn{2}{c|}{1}&\multicolumn{2}{c|}{0}&\multicolumn{2}{c|}{0}&\multicolumn{2}{c|}{0}&\multicolumn{2}{c|}{0}&\multicolumn{2}{c|}{0}&\multicolumn{2}{c}{0}\\

\bottomrule
\end{tabular}
}
\end{small}
\end{center}
\end{table}
\revision{
Our full results in few-shot forecasting tasks are detailed in \shortautoref{tab:few-shot-forecasting-10per-full} and \shortautoref{tab:few-shot-forecasting-5per-full}. Within the scope of 10\% few-shot learning, \method secures SOTA performance in \textbf{32} out of 35 cases, spanning seven different time series benchmarks. Our approach's advantage becomes even more pronounced in the context of 5\% few-shot scenarios, achieving SOTA results in \textbf{21} out of 32 cases. We attribute this to the successful knowledge activation in our reprogrammed LLM.
}

\vspace{-8mm}
\revision{\subsection{Zero-shot Forecasting}}
The full results of zero-shot forecasting are summarized in \shortautoref{tab:zero-shot-forecasting}. \revision{\method remarkably surpasses the six most competitive time series models in zero-shot adaptation.} Overall, we observe over \textbf{23.5\%} and \textbf{12.4\%} MSE and MAE reductions across all baselines on average. Our improvements are consistently significant on those typical cross-domain scenarios (e.g., ETTh2 $\rightarrow$ ETTh1 and ETTm2 $\rightarrow$ ETTm1), over \textbf{20.8\%} and \textbf{11.3\%} on average w.r.t. MSE and MAE. 
\revision{Significantly, \method exhibits superior performance gains in comparison to LLMTime~\citep{gruver2023large}, which employs a similarly sized backbone LLM (7B) and is the latest effort in leveraging LLMs for zero-shot time series forecasting.}
We attribute this success to our reprogramming framework being better at activating the LLM's knowledge transfer and reasoning capabilities in a resource-efficient manner when performing time series tasks.

\begin{table}[h!]
\begin{center}
\captionsetup{font=small} 
\caption{\revision{Full zero-shot learning results on ETT datasets. A lower value indicates better performance. \boldres{Red}: the best, \secondres{Blue}: the second best.}}
\label{tab:zero-shot-forecasting}
\begin{small}
\scalebox{0.75}{
\setlength\tabcolsep{3pt}
\begin{tabular}{c|c|cc|cc|cc|cc|cc|cc|cc}
\toprule
\multicolumn{2}{c|}{Methods}&\multicolumn{2}{c|}{\method}&\multicolumn{2}{c|}{\revision{LLMTime}}&\multicolumn{2}{c|}{GPT4TS}&\multicolumn{2}{c|}{DLinear}&\multicolumn{2}{c|}{PatchTST}&\multicolumn{2}{c|}{TimesNet}&\multicolumn{2}{c}{Autoformer}\\
\midrule
\multicolumn{2}{c|}{Metric} & MSE & MAE & \revision{MSE} & \revision{MAE} & MSE & MAE & MSE & MAE & MSE & MAE& MSE & MAE & MSE & MAE \\
\midrule
\multirow{5}{*}{\rotatebox{0}{$ETTh1$} $\rightarrow$ \rotatebox{0}{$ETTh2$}} & 
96  &\boldres{0.279} & \boldres{0.337} &0.510 &0.576 & 0.335 & 0.374 & 0.347 & 0.400 & \secondres{0.304} & \secondres{0.350} & 0.358 & 0.387 & 0.469 & 0.486 \\
& 192 & \boldres{0.351} & \boldres{0.374} &0.523 &0.586 & 0.412 & 0.417 & 0.447 & 0.460 & \secondres{0.386} & \secondres{0.400} & 0.427 & 0.429 & 0.634 & 0.567 \\
& 336 & \boldres{0.388} & \boldres{0.415} &0.640 &0.637 & 0.441 & 0.444 & 0.515 & 0.505 & \secondres{0.414} & \secondres{0.428} & 0.449 & 0.451 & 0.655 & 0.588 \\
& 720 & \boldres{0.391} & \boldres{0.420} &2.296 &1.034 & 0.438 & 0.452 & 0.665 & 0.589 & \secondres{0.419} & \secondres{0.443} & 0.448 & 0.458 & 0.570 & 0.549 \\
&Avg & \boldres{0.353} & \boldres{0.387} &0.992 &0.708 & 0.406 & 0.422 & 0.493 & 0.488 & \secondres{0.380} & \secondres{0.405} & 0.421 & 0.431 & 0.582 & 0.548 \\
\midrule
\multirow{5}{*}{\rotatebox{0}{$ETTh1 $} $\rightarrow$ \rotatebox{0}{$ETTm2 $}}
& 96  & \boldres{0.189} & \boldres{0.293} &0.646&0.563& 0.236 & 0.315 & 0.255 & 0.357 & \secondres{0.215} & \secondres{0.304} & 0.239 & 0.313 & 0.352 & 0.432 \\
& 192 & \boldres{0.237} & \boldres{0.312} &0.934&0.654& 0.287 & 0.342 & 0.338 & 0.413 & \secondres{0.275} & \secondres{0.339} & 0.291 & 0.342 & 0.413 & 0.460 \\
& 336 & \boldres{0.291} & \boldres{0.365} &1.157&0.728& 0.341 & 0.374 & 0.425 & 0.465 & \secondres{0.334} & \secondres{0.373} & 0.342 & 0.371 & 0.465 & 0.489 \\
& 720 & \boldres{0.372} & \boldres{0.390} &4.730 &1.531 & 0.435 & \secondres{0.422} & 0.640 & 0.573 & \secondres{0.431} & 0.424 & 0.434 & 0.419 & 0.599 & 0.551 \\
&Avg & \boldres{0.273} & \boldres{0.340} &1.867 &0.869 & 0.325 & 0.363 & 0.415 & 0.452 & \secondres{0.314} & \secondres{0.360} & 0.327 & 0.361 & 0.457 & 0.483 \\
\midrule
\multirow{5}{*}{\rotatebox{0}{$ETTh2 $} $\rightarrow$ \rotatebox{0}{$ETTh1 $}}
& 96  & \boldres{0.450} & \boldres{0.452} &1.130 &0.777 & 0.732 & 0.577 & 0.689 & 0.555 & \secondres{0.485} & \secondres{0.465} & 0.848 & 0.601 & 0.693 & 0.569 \\
& 192 & \boldres{0.465} & \boldres{0.461} &1.242 &0.820 & 0.758 & 0.559 & 0.707 & 0.568 &\secondres{0.565} & \secondres{0.509} & 0.860 & 0.610 & 0.760 & 0.601 \\
& 336 & \boldres{0.501} & \boldres{0.482} &1.328 &0.864 & 0.759 & 0.578 & 0.710 & 0.577 & \secondres{0.581} & \secondres{0.515} & 0.867 & 0.626 & 0.781 & 0.619 \\
& 720 & \boldres{0.501} & \boldres{0.502} &4.145 &1.461 & 0.781 & 0.597 & 0.704 & 0.596 & \secondres{0.628} & \secondres{0.561} & 0.887 & 0.648 & 0.796 & 0.644 \\
&Avg & \boldres{0.479} & \boldres{0.474} &1.961 &0.981 & 0.757 & 0.578 & 0.703 & 0.574 & \secondres{0.565} & \secondres{0.513} & 0.865 & 0.621 & 0.757 & 0.608 \\
\midrule
\multirow{5}{*}{\rotatebox{0}{$ETTh2 $} $\rightarrow$ \rotatebox{0}{$ETTm2 $}}
& 96  & \boldres{0.174} & \boldres{0.276} &0.646&0.563& 0.253 & 0.329 & 0.240 & 0.336 & \secondres{0.226} & \secondres{0.309} & 0.248 & 0.324 & 0.263 & 0.352 \\
& 192 & \boldres{0.233} & \boldres{0.315} &0.934&0.654& 0.293 & 0.346 & 0.295 & 0.369 & \secondres{0.289} & \secondres{0.345} & 0.296 & 0.352 & 0.326 & 0.389 \\
& 336 & \boldres{0.291} & \boldres{0.337} &1.157&0.728& 0.347 & \secondres{0.376} & \secondres{0.345} & 0.397 & 0.348 & 0.379 & 0.353 & 0.383 & 0.387 & 0.426 \\
& 720 & \boldres{0.392} & \boldres{0.417} &4.730 &1.531 & 0.446 & 0.429 & \secondres{0.432} & 0.442 &  0.439 & 0.427 & 0.471 & 0.446 & 0.487 & 0.478 \\
&Avg & \boldres{0.272} & \boldres{0.341} &1.867 &0.869 & 0.335 & 0.370 & 0.328 & 0.386 & \secondres{0.325} & \secondres{0.365} & 0.342 & 0.376 & 0.366 & 0.411 \\
\midrule
\multirow{5}{*}{\rotatebox{0}{$ETTm1 $} $\rightarrow$ \rotatebox{0}{$ETTh2 $}}
& 96  & \boldres{0.321} & \boldres{0.369} &0.510 &0.576 & \secondres{0.353} & 0.392 & 0.365 & 0.415 & 0.354 & \secondres{0.385} & 0.377 & 0.407 & 0.435 & 0.470 \\
& 192 & \boldres{0.389} & \boldres{0.410} &0.523 &0.586 & \secondres{0.443} & 0.437 & 0.454 & 0.462 & 0.447 & \secondres{0.434} & 0.471 & 0.453 & 0.495 & 0.489 \\
& 336 & \boldres{0.408} & \boldres{0.433} &0.640 &0.637 & \secondres{0.469} & \secondres{0.461} & 0.496 & 0.494 & 0.481 & 0.463 & 0.472 & 0.484 & 0.470 & 0.472 \\
& 720 & \boldres{0.406} & \boldres{0.436} &2.296 &1.034 & \secondres{0.466} & \secondres{0.468} & 0.541 & 0.529 & 0.474 & 0.471 & 0.495 & 0.482 & 0.480 & 0.485 \\
&Avg & \boldres{0.381} & \boldres{0.412} &0.992 &0.708 & \secondres{0.433} & 0.439 & 0.464 & 0.475 & 0.439 & \secondres{0.438} & 0.457 & 0.454 & 0.470 & 0.479 \\
\midrule
\multirow{5}{*}{\rotatebox{0}{$ETTm1 $} $\rightarrow$ \rotatebox{0}{$ETTm2 $}}
& 96  & \boldres{0.169} & \boldres{0.257} &0.646&0.563& 0.217 & 0.294 & 0.221 & 0.314 & \secondres{0.195} & \secondres{0.271} & 0.222 & 0.295 & 0.385 & 0.457 \\
& 192 & \boldres{0.227} & \boldres{0.318} &0.934&0.654& 0.277 & 0.327 & 0.286 & 0.359 & \secondres{0.258} & \secondres{0.311} & 0.288 & 0.337 & 0.433 & 0.469 \\
& 336 & \boldres{0.290} & \boldres{0.338} &1.157&0.728& 0.331 & 0.360 & 0.357 & 0.406 & \secondres{0.317} & \secondres{0.348} & 0.341 & 0.367 & 0.476 & 0.477 \\
& 720 & \boldres{0.375} & \boldres{0.367} &4.730 &1.531 & 0.429 & 0.413 & 0.476 & 0.476 & \secondres{0.416} & \secondres{0.404} & 0.436 & 0.418 & 0.582 & 0.535 \\
&Avg & \boldres{0.268} & \boldres{0.320} &1.867 &0.869 & 0.313 & 0.348 & 0.335 & 0.389 & \secondres{0.296} & \secondres{0.334} & 0.322 & 0.354 & 0.469 & 0.484 \\
\midrule
\multirow{5}{*}{\rotatebox{0}{$ETTm2 $} $\rightarrow$ \rotatebox{0}{$ETTh2 $}}
& 96  & \boldres{0.298} & \boldres{0.356} &0.510 &0.576 & 0.360 & 0.401 & 0.333 & 0.391 & \secondres{0.327} & \secondres{0.367} & 0.360 &  0.401 & 0.353 & 0.393 \\
& 192 & \boldres{0.359} & \boldres{0.397} &0.523 &0.586 & 0.434 & 0.437 & 0.441 & 0.456 & \secondres{0.411} & \secondres{0.418} & 0.434 & 0.437 & 0.432 & 0.437 \\
& 336 & \boldres{0.367} & \boldres{0.412} &0.640 &0.637 & 0.460 & 0.459 & 0.505 & 0.503 & \secondres{0.439} & \secondres{0.447} & 0.460 & 0.459 & 0.452 & 0.459 \\
& 720 & \boldres{0.393} & \boldres{0.434} &2.296 &1.034 & 0.485 & 0.477 & 0.543 & 0.534 & 0.459 & 0.470 & 0.485 & 0.477 & \secondres{0.453} & \secondres{0.467} \\
&Avg & \boldres{0.354} & \boldres{0.400} &0.992 &0.708 & 0.435 & 0.443 & 0.455 & 0.471 & 0.409 & 0.425 & 0.435 & 0.443 & 0.423 & 0.439 \\
\midrule
\multirow{5}{*}{\rotatebox{0}{$ETTm2 $} $\rightarrow$ \rotatebox{0}{$ETTm1 $}}
& 96  & \boldres{0.359} & \boldres{0.397} &1.179& 0.781& 0.747 & 0.558 & 0.570 & 0.490 & \secondres{0.491} & \secondres{0.437} & 0.747 & 0.558 & 0.735 & 0.576 \\
& 192 & \boldres{0.390} & \boldres{0.420} &1.327&0.846& 0.781 & 0.560 & 0.590 & 0.506  & \secondres{0.530} & \secondres{0.470} & 0.781 & 0.560 & 0.753 & 0.586 \\
& 336 & \boldres{0.421} & \boldres{0.445} &1.478&0.902& 0.778 & 0.578 & 0.706 & 0.567  & \secondres{0.565} & \secondres{0.497} & 0.778 & 0.578 & 0.750 & 0.593 \\
& 720 & \boldres{0.487} & \boldres{0.488} &3.749 &1.408 & 0.769 & 0.573 & 0.731 & 0.584 & \secondres{0.686} & \secondres{0.565} & 0.769 & 0.573 & 0.782 & 0.609 \\
&Avg & \boldres{0.414} & \boldres{0.438} &1.933 &0.984 & 0.769 & 0.567 & 0.649 & 0.537 & \secondres{0.568} & \secondres{0.492} & 0.769 & 0.567 & 0.755 & 0.591 \\
\bottomrule

\end{tabular}
}
\end{small}
\end{center}
\end{table}